
\chapter{引言}
\label{chap:introduction}

考虑到大多数用户并无\LaTeX{}使用经验,本模板将\LaTeX{}的复杂性尽可能地进行了封装,开放出简单的接口,以便于使用者可以轻易地使用。同时,对使用\LaTeX{}撰写论文所遇到的一些主要难题,如插入图片、文献索引等,进行了详细的说明,并提供了相应的代码样本,理解了上述问题后,对于初学者而言,使用此模板撰写其学文论文将不存在实质性的困难。所以,如果您是初学者,请不要直接放弃,因为同样作为初学者的我,十分明白让\LaTeX{}变得简单易用的重要性,而这正是本模板所体现的。

此中国科学院大学学位论文模板ucasthesis基于中科院数学与系统科学研究院吴凌云研究员的CASthesis模板发展而来。当前ucasthesis 模板满足最新的中国科学院大学学位论文撰写要求和封面设定。模板兼顾不同操作系统 (Windows,Linux,Mac OS) 和\LaTeX{}编译引擎(pdflatex,xelatex,lualatex),完美地支持中文书签、中文渲染、中文粗体显示、拷贝PDF中的文本到其他文本编辑器等特性,此外,对模板的文档结构进行了精心设计,撰写了编译脚本提高模板的易用性和使用效率。

宏包的目的是简化学位论文的撰写,模板文档的默认设定是十分规范的,从而论文作者可以将精力集中到论文的内容上,而不需要在版面设置上花费精力。 同时,在编写模板的\LaTeX{}文档代码过程中,作者对各结构和命令进行了十分详细的注解,并提供了整洁一致的代码结构,对文档的仔细阅读可以为初学的您提供一个学习\LaTeX{}的窗口。除此之外,整个模板的架构十分注重通用性,事实上,本模板不仅是中国科学院大学学文论文模板,同时,也是使用\LaTeX{}撰写中英文Article或Book的通用模板,并为使用者的个性化设定提供了接口和相应的代码。

\section{系统要求}\label{sec:system}

\href{https://github.com/mohuangrui/ucasthesis}{\texttt{ucasthesis}} 宏包可以在目前主流的 \href{https://en.wikibooks.org/wiki/LaTeX/Introduction}{\LaTeX{}} 编译系统中使用,例如C\TeX{}套装 (请勿混淆C\TeX{}套装与ctex宏包。C\TeX{}套装是集成了许多\LaTeX{}组件的\LaTeX{}编译系统,因已停止维护,\textbf{不再建议使用}。 \href{https://ctan.org/pkg/ctex?lang=en}{ctex} 宏包就如同ucasthesis一样,是\LaTeX{}命令集,其维护状态非常活跃,并被所有主流的\LaTeX{}编译系统所默认集成,是几乎所有\LaTeX{}中文文档的核心架构。)、MiK\TeX{}、\TeX{}Live。推荐的 \href{https://en.wikibooks.org/wiki/LaTeX/Installation}{\LaTeX{}编译系统} 和 \href{https://en.wikibooks.org/wiki/LaTeX/Installation}{\LaTeX{}文本编辑器} 为
\begin{center}
    %\footnotesize% fontsize
    %\setlength{\tabcolsep}{4pt}% column separation
    %\renewcommand{\arraystretch}{1.5}% row space 
    \begin{tabular}{lcc}
        \hline\hline
        %\multicolumn{num_of_cols_to_merge}{alignment}{contents} \\
        %\cline{i-j}% partial hline from column i to column j
        操作系统 & \LaTeX{}编译系统 & \LaTeX{}文本编辑器\\
        \hline
        Linux & \href{https://www.tug.org/texlive/acquire-netinstall.html}{\TeX{}Live Full} & \href{http://www.xm1math.net/texmaker/}{Texmaker}\\
        MacOS & \href{https://www.tug.org/mactex/}{Mac\TeX{} Full} & \href{http://www.xm1math.net/texmaker/}{Texmaker}\\
        Windows & \href{https://www.tug.org/texlive/acquire-netinstall.html}{\TeX{}Live Full} 或 \href{https://miktex.org/download}{Mik\TeX{}} & \href{http://www.xm1math.net/texmaker/}{Texmaker}\\
        \hline\hline
    \end{tabular}
\end{center}

\LaTeX{}编译系统 (如MiK\TeX{}、\TeX{}Live) 用于提供编译环境,\LaTeX{}文本编辑器 (如Texmaker) 用于编辑\TeX{}源文件。请用户一定从上述各软件的官网下载安装程序,勿使用其它程序源。\textbf{\LaTeX{}编译系统和\LaTeX{}编辑器分别安装成功后,用户即完成了\LaTeX{}的系统配置},无需其他手动干预和配置。若用户的系统原带有旧版的\LaTeX{}编译系统并想安装新版,其\textbf{先卸载干净旧版再安装新版}。

\section{问题反馈}

关于\LaTeX{}知识类的问题,请在网上搜索并阅读相关的\LaTeX{}教程,特别是 \href{https://en.wikibooks.org/wiki/LaTeX}{\LaTeX{} Wikibook}。

若遇到关于模板编译和功能类的问题,请先阅读 \href{https://github.com/mohuangrui/ucasthesis}{位于主页底部的用户指南}。若问题仍无法得到解决,请将问题反馈至 \href{https://github.com/mohuangrui/ucasthesis/issues}{Github/ucasthesis/issues} 或是发送邮件到

\begin{center}
莫晃锐 (mohuangrui) \quad mohuangrui@gmail.com
\end{center}

欢迎大家反馈模板不足之处,一起不断改进模板。希望大家向同事积极推广\LaTeX{},一起更高效地做科研。

\section{模板下载}

\begin{center}
    \href{https://github.com/mohuangrui/ucasthesis}{Github/ucasthesis}: \url{https://github.com/mohuangrui/ucasthesis}
\end{center}
