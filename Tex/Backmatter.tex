\chapter{致\quad 谢}

值此论文完成之际,谨在此向多年来给予我关心和帮助的老师、学长、同学、
朋友和家人表示衷心的感谢!

没有~ctex package~的众多前辈的辛勤付出和~CASthesis package~作者吴凌云学长的贡献,
~\LaTeX{}~菜鸟的我是无法完成此学位论文模板的。在~\LaTeX{}~中的一点一滴的成长源于
开源社区的众多资料和教程,在此对所有前辈们的付出表示感谢!

......

谨把本文献给我最敬爱的父亲!


\rightline{张三}
\rightline{2018年6月}


\chapter{作者简历及攻读学位期间发表的学术论文与研究成果}

\section*{作者简历}

\subsection*{casthesis作者}

吴凌云,男,福建省屏南县人,1975 年出生,中国科学院数学与系统科学研究院博士研究生。

通讯地址:北京市~2734 信箱,中科院数学与系统科学研究院应用数学所

邮编:100080

E-mail: aloft@ctex.org

\subsection*{ucasthesis作者}

莫晃锐,男,湖南省湘潭县人,1989 年出生,中国科学院力学研究所硕士研究生。

通讯地址:北京市北四环西路15号中国科学院力学研究所

邮编:100190

E-mail: mohuangrui@gmail.com

\begin{description}
 \item[获奖情况:] XXX
 \item[\quad\quad\quad\quad\quad] XXX
 \item[\quad\quad\quad\quad\quad] XXX
\end{description}


\section*{已发表(或正式接受)的学术论文:}

\noindent
一作论文:
\setcounter{mycnt}{0}

\noindent
\stepaddpaper \hangindent=0.7cm\hangafter=1 Stamerjohanns H, Ginev D, David C, et al. MathML-aware article conversion from LaTeX [J]. Towards a Digital Mathematics Library, 2009, 16(2): 109–120.

\noindent
\stepaddpaper \hangindent=0.7cm\hangafter=1 Thesis Template of the University of Chinese Academy of Sciences, 2014.


\noindent
合作论文:
\setcounter{mycnt}{0}
\newline\noindent
\stepaddpaper 作者发表的论文.
\newline\noindent
\stepaddpaper 作者发表的论文.
\newline\noindent
\stepaddpaper XXXX.


\section*{申请或已获得的专利:}

(无专利时此项不必列出)

\section*{参加的研究项目及获奖情况:}

可以随意添加新的条目或是结构



