\chapter{致\quad 谢}

值此论文完成之际,谨在此向多年来给予我关心和帮助的老师、学长、同学、
朋友和家人表示衷心的感谢!

没有~ctex package~的众多前辈的辛勤付出和~CASthesis package~作者吴凌云学长的贡献,
~\LaTeX{}~菜鸟的我是无法完成此学位论文模板的。在~\LaTeX{}~中的一点一滴的成长源于
开源社区的众多资料和教程,在此对所有前辈们的付出表示感谢!

......

谨把本文献给我最敬爱的父亲!


\rightline{张三}
\rightline{2018年6月}


\chapter{作者简历及攻读学位期间发表的学术论文与研究成果}

\section*{作者简历:}

\noindent
作者简历应包括从大学起到申请学位时的个人学习工作经历,如:

\noindent
××××年××月——××××年××月,在××大学××院(系)获得学士学位。

\noindent
××××年××月——××××年××月,在××大学××院(系)获得硕士学位。

\noindent
××××年××月——××××年××月,在中国科学院××研究所(或中国科学院大学××院系)攻读博士/硕士学位。

\noindent
获奖情况: 2017年获得中国科学院院长优秀奖

\noindent
\quad\quad\quad\quad\quad 2016年获得研究生国家奖学金

\noindent
\quad\quad\quad\quad\quad 2012年和2015年获得中国科学院大学三好学生



\section*{已发表(或正式接受)的学术论文:}

\noindent
一作论文:
\setcounter{mycnt}{0}

\noindent
\stepaddpaper  Stamerjohanns H, Ginev D, David C, et al. MathML-aware article conversion from LaTeX [J]. Towards a Digital Mathematics Library, 2009, 16(2): 109–120.

\noindent
\stepaddpaper  Thesis Template of the University of Chinese Academy of Sciences, 2014.


\noindent
合作论文:

\setcounter{mycnt}{0}

\noindent
\stepaddpaper 作者发表的论文.

\noindent
\stepaddpaper 作者发表的论文.

\noindent
\stepaddpaper XXXX.


\section*{申请或已获得的专利:}
\noindent
(无专利时此项不必列出)

\section*{参加的研究项目及获奖情况:}
\noindent
可以随意添加新的条目或是结构



